
\begin{table}[t]

\caption{\label{tab:print-table-hyp}Table 1. Hypotheses for covariates affecting landings.  See text for details.}
\centering
\begin{tabular}{>{\raggedright\arraybackslash}p{1cm}|>{\raggedright\arraybackslash}p{.5cm}|>{\raggedright\arraybackslash}p{8.5cm}|>{\raggedright\arraybackslash}p{2cm}|>{\raggedright\arraybackslash}p{3cm}}
\hline
Stage & Time & Hypothesis & Responses & Covariates\\
\hline
Age 2+ & Jun-Sep & HDD1.  Q3 catch is dominated by mature age 2+ fish, thus abundance of the 1-yr and 2-yr ages in the prior season should be correlated with the abundance of spawners this year. & Catch Q3 year t & Catch Q4 year t-1 and Q1-2 year t\\
\hline
Age 1-2 & Oct-May & HDD2.  Abundance of 1-yr and 2-yr fish should be correlated with strength of the cohorts from the previous two seasons.  The catch in Q3 is dominated by mature fish, thus catch in Q3 in the prior two years is expected to be correlated with post-monsoon catch. & Catch Q4 year t and Q1-2 year t+1 & Catch Q3 year t-1 and year t-2\\
\hline
Age 2+ & Oct-May & HDD3.  Because age 2 fish also appear in the post-monsoon catch, we also expect the post-monsoon catch in the previous season to be correlated with the post-monsoon catch in the current season. & Catch Q4 year t and Q1-2 year t+1 & Catch Q4 year t-1 and Q1-2 year t\\
\hline
Spawn & Jun-Sep & HS1.  The onset of monsoon precipitation triggers movement of adults from offshore to spawning areas due to changes in salinity, turbulence or noise. Spent adults migrate inshore and are exposed to the fishery. & Catch Q3 year t & Seasonal precipitation anomaly during Jun-Jul in year t\\
\hline
 &  & HS2.  The level of precipitation in pre-monsoon predicts spawning strength. & Catch Q3 year t & Seasonal precipitation anomaly during Apr-May in year t\\
\hline
 &  & HS3.  Low SST is associated with delayed and limited spawning [@JacobsonMacCall1995] as a behavioral response by adults to avoid exposing larvae to low temperatures associated with poor survival. & Catch Q3 year t & Average SST during Jun-Sep in year t\\
\hline
 &  & HS4.  Extremely high upwelling brings poorly oxygenated water to the surface causing sardines to move offshore [@Pillai1991]. & Catch Q3 year t & Average upwelling index Jun-Sep, Max upwelling index Jun-Sep in year t\\
\hline
 &  & HS5. Salinity changes due to precipitation or river run-off trigger spawning.  After spawning, spent adult move to inshore waters and are exposed to fishery. & Catch Q3 year t & Average Salinity during Jun-Sep in year t\\
\hline
\end{tabular}
\end{table}

\section{}\label{section}

\clearpage

\begin{table}[t]

\caption{\label{tab:unnamed-chunk-3}Table 1. Continued.}
\centering
\begin{tabular}{>{\raggedright\arraybackslash}p{1cm}|>{\raggedright\arraybackslash}p{.5cm}|>{\raggedright\arraybackslash}p{8.5cm}|>{\raggedright\arraybackslash}p{2cm}|>{\raggedright\arraybackslash}p{3cm}}
\hline
Stage & Time & Hypothesis & Responses & Covariates\\
\hline
Larvae & Jul-Oct & HL1.  Larval mortality is higher in colder water due to low motility causing increased predation and slower somatic growth. Low SST is also associated with strong upwelling which advects larvae into offshore waters. & Catch Q4 year t and Q1-2 year t+1 & Average SST during Jun-Sep, Cum DD Jun-Sep in year t-1\\
\hline
Larvae &  & HL2. Extremely strong upwelling brings poorly oxygenated water to the surface causing larval mortality and advects larvae offshore. & Catch Q4 year t and Q1-2 year t+1 & Ave. upwelling index Jun-Sep, max upwelling index Jun-Sep in year t-1\\
\hline
Juv. &  & HJ1. Upwelling is associated with higher productivity and higher density of zooplankton, which leads to better larval and juvenile growth and survival.  Thus the strength of upwelling during the monsoon should be associated with higher biomass in subsequent years. & Catch Q4 year t and Q1-2 year t+1 & Ave. upwelling index Jun-Sep, max upwelling index Jun-Sep in year t-1 and t-2\\
\hline
Juv. &  & HJ2. Chlorophyll blooms are signatures of high productivity from nutrient influx either due to upwelling or coastal inputs.  Bloom intensity in prior years should be associated with future sardine biomass. & Catch Q4 year t and Q1-2 year t+1 & Ave. Chl-a density Jun-Dec, Chl-a density Jun-Dec in year t-1 and t-2\\
\hline
All ages & Mar-Apr & HA1. During the Mar-Apr, the sea temperatures are high and sardines migrate offshore to avoid high temp. & Catch Q2 year t & Ave. SST Q2 year t, max SST Q2 year t\\
\hline
\end{tabular}
\end{table}



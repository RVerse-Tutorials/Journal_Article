\documentclass[]{article}
\usepackage{lmodern}
\usepackage{amssymb,amsmath}
\usepackage{ifxetex,ifluatex}
\usepackage{fixltx2e} % provides \textsubscript
\ifnum 0\ifxetex 1\fi\ifluatex 1\fi=0 % if pdftex
  \usepackage[T1]{fontenc}
  \usepackage[utf8]{inputenc}
\else % if luatex or xelatex
  \ifxetex
    \usepackage{mathspec}
  \else
    \usepackage{fontspec}
  \fi
  \defaultfontfeatures{Ligatures=TeX,Scale=MatchLowercase}
\fi
% use upquote if available, for straight quotes in verbatim environments
\IfFileExists{upquote.sty}{\usepackage{upquote}}{}
% use microtype if available
\IfFileExists{microtype.sty}{%
\usepackage{microtype}
\UseMicrotypeSet[protrusion]{basicmath} % disable protrusion for tt fonts
}{}
\usepackage[margin=1in]{geometry}
\usepackage{hyperref}
\hypersetup{unicode=true,
            pdfborder={0 0 0},
            breaklinks=true}
\urlstyle{same}  % don't use monospace font for urls
\usepackage{graphicx,grffile}
\makeatletter
\def\maxwidth{\ifdim\Gin@nat@width>\linewidth\linewidth\else\Gin@nat@width\fi}
\def\maxheight{\ifdim\Gin@nat@height>\textheight\textheight\else\Gin@nat@height\fi}
\makeatother
% Scale images if necessary, so that they will not overflow the page
% margins by default, and it is still possible to overwrite the defaults
% using explicit options in \includegraphics[width, height, ...]{}
\setkeys{Gin}{width=\maxwidth,height=\maxheight,keepaspectratio}
\IfFileExists{parskip.sty}{%

\begin{table}[t]

\caption{\label{tab:print-table-hyp}Table 1. Hypotheses for covariates affecting landings.  $S_t$ is quarter 3 (July-September) catch in the current season, $S_{t-1}$ is quarter 3 catch in the previous season. $N_t$ is the post-monsoon October-March catch in the current season and $N_{t-1}$ is the October-March catch in the prior season. Because the fishing season is July-June, $N_t$ spans two calendar years.}
\centering
\begin{tabular}{>{\raggedright\arraybackslash}p{1cm}|>{\raggedright\arraybackslash}p{8.5cm}|>{\raggedright\arraybackslash}p{1cm}|>{\raggedright\arraybackslash}p{3cm}}
\hline
Stage & Hypothesis & Resp. & Covariates\\
\hline
Age 2+ & DD1.  $S_t$ is dominated by mature age 2+ fish, thus abundance of the 1-yr and 2-yr ages in the prior season (Oct-Mar catch) should be correlated with the abundance of mature fish this year. & $S_t$ & $N_{t-1}$\\
\hline
Age 1-2 & DD2.  Abundance of 1-yr and 2-yr fish should be correlated with strength of the cohorts from the previous two seasons.  The quarter 3 catch, dominated by mature fish, in the prior two years is expected to be correlated with post-monsoon catch. & $N_t$ & $S_{t-1}$ and $S_{t-2}$\\
\hline
Age 2+ & DD3.  Because age 2 fish appear in the post-monsoon catch, we also expect the post-monsoon catch (dominated by age 1 and 2) in the previous season to be correlated with the post-monsoon catch in the current season. Post-monsoon catch two seasons prior should be minimally correlated with current post-monsoon catch. & $N_t$ & $N_{t-1}$\\
\hline
Spawn & S1.  The onset of monsoon precipitation triggers movement of adults from offshore to spawning areas due to changes in salinity, turbulence or noise. Spent adults migrate inshore and are exposed to the fishery. & $S_t$ & Seasonal precipitation during Jun-Jul in year t\\
\hline
Spawn & S2.  The level of precipitation in pre-monsoon months predicts spawning strength. & $S_t$ & Seasonal precipitation during Apr-May in year t\\
\hline
Spawn & S3.  Extremely high upwelling brings poorly oxygenated water and very low temperatures to the surface causing a delayed or poor spawning while optimal upwelling conditions leads to increased spawing. Spawning affects the recruitment of 0-age fish to the fishery and thus the post-monsoon catch. & $N_t$ and $S_t$ & Ave. Jun-Sep upwelling index in year $t$\\
\hline
Spawn & S4.  Extreme heat events in the pre-spawning months cause mature fish to move offshore away from productive feeding areas leading to poor spawning condition. Poor recruitment leads to low 0-age in current season catch and 1-age fish in next season's catch. & $N_t$ & Ave. nearshore Mar-May SST in year $t$ and $t-1$\\
\hline
\end{tabular}
\end{table}

\clearpage

\begin{table}[t]

\caption{\label{tab:unnamed-chunk-3}Table 1. Continued.}
\centering
\begin{tabular}{>{\raggedright\arraybackslash}p{1cm}|>{\raggedright\arraybackslash}p{8.5cm}|>{\raggedright\arraybackslash}p{1cm}|>{\raggedright\arraybackslash}p{3cm}}
\hline
Stage & Hypothesis & Resp. & Covariates\\
\hline
Larval & L1.  The prior year post-monsoon larval survival and growth is associated with higher future biomass. Larval growth and survival is highest in an intermediate temperature window. Low SST at this time is also indicative of strong upwelling which advects larvae into offshore waters where productivity is lower. & $N_t$ and $S_t$ & Average nearshore SST during Oct-Dec in year t-1\\
\hline
Larval & L2. Upwelling is associated with higher productivity and higher density of zooplankton, which leads to better larval and juvenile growth and survival.  The strength of summer upwelling should be associated with higher biomass in future years. However, extremely strong upwelling brings poorly oxygenated water to the surface causing larval mortality and advects larvae offshore. & $N_t$ and $S_t$ & Ave. Jun-Sep upwelling index in year t-1 and t-2\\
\hline
Larval & L3. Chlorophyll blooms are signatures of high productivity from nutrient influx either due to upwelling or coastal inputs.  The monsoon bloom intensity in prior years should be associated with 0-year fish abundance in year $t$ and future sardine biomass. & $N_t$ and $S_t$ & Ave. Chl-a density Jun-Sep in year $t$, $t-1$ and $t-2$\\
\hline
All ages & A1. The changes brought about by the El Niño Southern Oscillation (ENSO) cycle have a variety of effects on environmental parameters (precipitation, SST, thermal fronts, Wind) which impacts spawning and early survival. This in turn impacts the overall abundance. & $N_t$ and $S_t$ & ONI year t-1\\
\hline
\end{tabular}
\end{table}

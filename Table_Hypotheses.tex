\PassOptionsToPackage{unicode=true}{hyperref} % options for packages loaded elsewhere
\PassOptionsToPackage{hyphens}{url}
%
\documentclass[]{article}
\usepackage{lmodern}
\usepackage{amssymb,amsmath}
\usepackage{ifxetex,ifluatex}
\usepackage{fixltx2e} % provides \textsubscript
\ifnum 0\ifxetex 1\fi\ifluatex 1\fi=0 % if pdftex
  \usepackage[T1]{fontenc}
  \usepackage[utf8]{inputenc}
  \usepackage{textcomp} % provides euro and other symbols
\else % if luatex or xelatex
  \usepackage{unicode-math}
  \defaultfontfeatures{Ligatures=TeX,Scale=MatchLowercase}
\fi
% use upquote if available, for straight quotes in verbatim environments
\IfFileExists{upquote.sty}{\usepackage{upquote}}{}
% use microtype if available
\IfFileExists{microtype.sty}{%
\usepackage[]{microtype}
\UseMicrotypeSet[protrusion]{basicmath} % disable protrusion for tt fonts
}{}
\IfFileExists{parskip.sty}{%
\usepackage{parskip}
}{% else
\setlength{\parindent}{0pt}
\setlength{\parskip}{6pt plus 2pt minus 1pt}
}
\usepackage{hyperref}
\hypersetup{
            pdfborder={0 0 0},
            breaklinks=true}
\urlstyle{same}  % don't use monospace font for urls
\usepackage[margin=1in]{geometry}
\usepackage{graphicx,grffile}
\makeatletter
\def\maxwidth{\ifdim\Gin@nat@width>\linewidth\linewidth\else\Gin@nat@width\fi}
\def\maxheight{\ifdim\Gin@nat@height>\textheight\textheight\else\Gin@nat@height\fi}
\makeatother
% Scale images if necessary, so that they will not overflow the page
% margins by default, and it is still possible to overwrite the defaults
% using explicit options in \includegraphics[width, height, ...]{}
\setkeys{Gin}{width=\maxwidth,height=\maxheight,keepaspectratio}
\setlength{\emergencystretch}{3em}  % prevent overfull lines
\providecommand{\tightlist}{%
  \setlength{\itemsep}{0pt}\setlength{\parskip}{0pt}}
\setcounter{secnumdepth}{0}
% Redefines (sub)paragraphs to behave more like sections
\ifx\paragraph\undefined\else
\let\oldparagraph\paragraph
\renewcommand{\paragraph}[1]{\oldparagraph{#1}\mbox{}}
\fi
\ifx\subparagraph\undefined\else
\let\oldsubparagraph\subparagraph
\renewcommand{\subparagraph}[1]{\oldsubparagraph{#1}\mbox{}}
\fi

% set default figure placement to htbp
\makeatletter
\def\fps@figure{htbp}
\makeatother

\usepackage[labelformat=empty]{caption}
\usepackage{booktabs}
\usepackage{longtable}
\usepackage{array}
\usepackage{multirow}
\usepackage{wrapfig}
\usepackage{float}
\usepackage{colortbl}
\usepackage{pdflscape}
\usepackage{tabu}
\usepackage{threeparttable}
\usepackage{threeparttablex}
\usepackage[normalem]{ulem}
\usepackage{makecell}
\usepackage{xcolor}

\author{}
\date{\vspace{-2.5em}}

\begin{document}

\renewcommand{\arraystretch}{1.2}

\begin{table}

\caption{\label{tab:print-table-hyp}Table 1. Hypotheses for covariates affecting landings.  The left column shows the model tested. The value to the left of $\sim$ is the response: $S_t$ Jul-Sep catch or $N_t$ Oct-Mar catch in the current season. To the right of $\sim$ are the explanatory variables. $S_{t-1}$ is Jul-Sep catch and $N_{t-1}$ is the Oct-Mar catch in the prior season. DD = hypotheses related to density-dependence. S = hypotheses related to catch during the spawning months. L = hypotheses related to larval and juvenile growth and survival. A = hypotheses affecting all ages. See the introduction for discussion of the literature on the effects of environmental covariates on sardine landings. ns-SST = nearshore SST. r-SST = regional (nearshore and offshore) SST. $t$, $t-1$, and $t-2$ subscripts indicate current, prior, and two seasons prior. ONI and DMI are ENSO indices described in the methods.}
\centering
\begin{tabular}[t]{>{\raggedright\arraybackslash}p{5.5cm}|>{\raggedright\arraybackslash}p{10cm}}
\hline
Hypothesis & Description\\
\hline
DD1 \newline $S_t \sim N_{t-1}$ & $S_t$ is age 1+ age fish and reflects the age 0-2 fish in $N_{t-1}$ which have aged 3-6 months (Nair et al. 2016).\\
\hline
DD2 \newline $S_t \sim S_{t-1} + S_{t-2}$ \newline $N_t \sim S_{t-1} + S_{t-2}$ & $S_t$ is age 1+ fish and is dominated by spent post-spawning fish (Nair et al. 2016). $S_t$ should be correlated with the abundance of the spawning stock and cohort strength. If cohort strength persists over time, landings in the current season should correlate with Jul-Sep landings in the previous two seasons.\\
\hline
DD3 \newline $N_t \sim N_{t-1}$ & $N_{t-1}$ includes age 0, 1, 2 and 2+ fish (Nair et al. 2016). Thus the age 0, 1, and 2 fish will appear 1 year later in the $N_t$ landings.\\
\hline
S1 \newline $S_t \sim$ Jun-Jul precipitation in $t$ & The magnitude of precipitation in June-July directly or indirectly signals mature fish to spawn offshore, after which the spent adults migrate inshore to the fishery (Anthony Raja 1969, 1974; Srinath 1998).\\
\hline
S2 \newline $S_t \sim$ Apr-Mar precipitation in $t$ & If there is pre-monsoon rain during Apr-May then spawning begins earlier and spent adults return to the fishery sooner and are exposed to the fishery longer.\\
\hline
S3 \newline $N_t \sim$ Apr-Mar precipitation in $t$ & Precipitation is an indicator of climatic conditions during the period of egg development. This affects the success of spawning, thus affecting the 0-year class and the Oct-Mar catch (Antony Raja 1969, 1974; Srinath 1998).\\
\hline
S4 \newline $S_t \sim$ Jun-Sep UPW in $t$ & High rates of upwelling (UPW) drive mature fish further offshore and leading to lower exposure to the fishery.  This movement may be driven by advection of high phytoplankton biomass offshore or upwelling of hypoxic water to the surface (Murty and Edelman 1970, Antony Raja 1973, Pillai et al. 1991).\\
\hline
S5 \newline $S_t \sim$ Mar-May r-SST in $t$ \newline $N_t \sim$ Mar-May r-SST in $t$ & Extreme heating events prior to the monsoon drive mature fish from the spawning areas, resulting in poor recruitment and fewer age 0 fish in the Oct-Mar catch (Antony Raja 1973, Pillai et al 1991).\\
\hline
\end{tabular}
\end{table}

\begin{table}

\caption{\label{tab:unnamed-chunk-2}Table 1. Continued.}
\centering
\begin{tabular}[t]{>{\raggedright\arraybackslash}p{5.5cm}|>{\raggedright\arraybackslash}p{10cm}}
\hline
Hypothesis & Description\\
\hline
L1 \newline $S_t \sim$ Oct-Dec ns-SST $t-1$ \newline $N_t \sim$ Oct-Dec ns-SST $t-1$ & Larval and juvenile growth and survival is affected by temperature. SST in the critical post-monsoon window, when somatic growth of the age 0 fish is highest, should be correlated with future abundance.\\
\hline
L2 \newline $S_t \sim$ Jun-Sep UPW in $t-1$ \& $t$ \newline $N_t \sim$ Jun-Sep UPW in $t-1$ \& $t$ & Higher upwelling rates lead to greater phytoplankton productivity, better larval and juvenile growth, and higher future landings. However, extreme upwelling may decrease catch due to hypoxic conditions or advection of phytoplankton biomass.\\
\hline
L3 \newline $S_t\sim$ CHL in $t-1$ \& $t$ \newline $N_t \sim$ CHL in $t-1$ \& $t$ & Surface chlorophyll-a (CHL) is a proxy for phytoplankton abundance and thus food availability, supporting greater fish abundance, and catch, in the current and future years.\\
\hline
A1 \newline $S_t \sim$ 2.5-yr ave. ns-SST \newline $N_t \sim$ 2.5-yr ave. ns-SST & Spawning, early survival, and recruitment of sardines depends on a multitude of cascading factors summarized by the multi-year average  nearshore SST. The multi-year average SST has been correlated with recruitment and abundance in other sardines (Checkley et al. 2017) and is related to the optimal temperature windows for sardines (Takasuka et al. 2007).\\
\hline
A2 \newline $S_t \sim$ ONI in $t-1$ \newline $N_t \sim$ ONI in $t-1$ & The El Niño/Southern Oscillation (ENSO) impacts a range of environmental parameters (e.g., precipitation, SST, frontal zones, winds, etc.) that may impact spawning and early survival of sardines (Supraba et al. 2016, Rohit et al. 2018).\\
\hline
A3 \newline $S_t \sim$ DMI in $t-1$ \newline $N_t \sim$ DMI in $t-1$ \& $t$ & Negative Dipole Mode Index (DMI) values in Sep-Nov are associated with anoxic events on the Kerala coast while positive values are associated with an absence of such events (Vallivattathillam et al. 2017).\\
\hline
\end{tabular}
\end{table}

\end{document}
